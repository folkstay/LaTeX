\documentclass[fullscreen=true,unicode,bookmarks=false]{beamer}
% Standard packages
\usepackage[utf8]{inputenc}
\usepackage[english,russian]{babel}
\usepackage{amsmath,amsfonts,amssymb}
\usepackage[T2A,T1]{fontenc}
\usepackage{wrapfig}
\usepackage{graphicx} % Графика
\setbeamertemplate{caption}[numbered]
\usepackage{hyperref} % Навигация по ссылкам в документе % Продвинутое цитирование
%\usepackage{cite}
\usepackage{color} %% это для отображения цвета в коде
\usepackage{listings} %% собственно, это и есть пакет listings
\usepackage{bm}
\usepackage{floatflt}
\usepackage{caption}
\mode<presentation>
%{
% % or ...

% % or whatever (possibly just delete it)
%}

% Setup appearance:
\usetheme{Singapore}
%\usetheme{Rochester}
%\usetheme{Frankfurt}
%\usetheme{Darmstadt}
%\usetheme{Warsaw}
%\usetheme{Boadilla}
\usecolortheme{seahorse}

%\useinnertheme[shadow]{rounded}
\usepackage{tikz}
\usepgflibrary{arrows}
\usepackage{graphicx}

\definecolor{MidnightBlue}{rgb}{0.2,0.2,0.7}
\definecolor{myblue}{rgb}{0.2,0.5,0.9}
\definecolor{lightyellow}{rgb}{1,1,0.7}
\definecolor{lightblue}{rgb}{0.7,0.8,1}
\definecolor{lightgreen}{rgb}{0.6,1,0.6}
\definecolor{darkgreen}{rgb}{0,0.5,0}
\definecolor{greenyellow}{rgb}{0.8,1,0.6}
\definecolor{ellipsecolor}{rgb}{0.8,1,0.8}
\definecolor{lightgray}{rgb}{0.7,0.7,0.7}
\definecolor{myred}{rgb}{0.8,0.2,0.2}
\definecolor{mypink}{rgb}{0.97,0.90,0.93} 

\tikzstyle{thickr}=[thick, myred]

% Animation
\newdimen\offset 


%\setbeamercovered{transparent}
\setbeamertemplate{navigation symbols}{}
%\usefonttheme[onlylarge]{structurebold}
%\setbeamerfont*{frametitle}{size=\normalsize,series=\bfseries}

% New commands
\newcommand{\R}{\mathbb{R}} %Set of real numbers
\newcommand{\N}{\mathbb{N}} %Set of natural numbers
\newcommand\op[1]{\mathop{\rm #1}\nolimits}
\newcommand\conv{\op{conv}}
\newcommand\len{\op{len}}
\newcommand\ext{\op{ext}}
\renewcommand{\le}{\leqslant} 
\renewcommand{\ge}{\geqslant} 

\newtheorem{mytheorem}{Theorem}
\newtheorem{mylemma}{Lemma}
\newtheorem{mycorollary}{Corollary}
\theoremstyle{example}
\newtheorem{myexample}{Example}
\theoremstyle{remark}
\newtheorem{remark}{Remark}
\theoremstyle{definition}
\newtheorem{mydefinition}{Definition}


\newcommand{\vo}{{\rm vol}}
\newcommand{\ver}{{\rm ver}}


\newcommand{\eps}{\varepsilon}
\newtheorem{Def}{Определение}
\newtheorem{Th1}{Теорема}

\setbeamercolor{section in head/foot}{bg=mypink,fg=white}
\setbeamercolor{palette primary}{bg=mypink,fg=black}
%\setbeamercolor{palette secondary}{bg=darkgreen,fg=white}
%\setbeamercolor{palette tertiary}{bg=MidnightBlue,fg=white}
%\setbeamercolor{palette quaternary}{bg=myblue,fg=white}

\setbeamercolor{itemize item}{fg=myred} % цвет кружочков первого уровня
\setbeamercolor{itemize subitem}{fg=darkgreen} % цвет кружочков второго уровня
\setbeamercolor{itemize subsubitem}{fg=myblue} % цвет кружочков третьего уровня


\title[Indie Web: Как вернуть себе интернет]
{Indie Web: Как вернуть себе интернет\\От платформ — к личному пространству.
\vspace{4mm}
\textbf{М.\,Д. Довгань}\\
folkstay@mail.ru\\
ЯрГУ им. П.\,Г. Демидова} 


\author{}

\institute{}
\date{} 
% \logo{\includegraphics[height=5mm]{images/logo.png}\vspace{-7pt}}

\begin{document}
%%%%%%%%%%%%%%%%%%%%%%%%%%%%%%%%%%%%%%%%%%%%%%%%%%%%%%%%%%%%%%%%%%%%%%%%%
% титульный слайд
\begin{frame}
\titlepage
\begin{flushright}
Преподаватель И.Н. Маслеников\\
\end{flushright}
\begin{center}
10 декабря 2025
\end{center}
\end{frame} 
%%%%%%%%%%%%%%%%%%%%%%%%%%%%%%%%%%%%%%%%%%%%%%%%%%%%%%%%%%%%%%%%%%%%%%%%%

%%%%%%%%%%%%%%%%%%%%%%%%%%%%%%%%%%%%%%%%%%%%%%%%%%%%%%%%%%%%%%%%%%%%%%%%%
\begin{frame}{Вопросы}

\begin{itemize}
\item Что не так с современным интернетом?
\item Что такое Indie Web
\item Преимущества Indie Web
\item Как начать? 
\item Neocities
\item Как распространяются сайты?

\end{itemize}



\end{frame}
%%%%%%%%%%%%%%%%%%%%%%%%%%%%%%%%%%%%%%%%%%%%%%%%%%%%%%%%%%%%%%%%%%%%%%%%%


%%%%%%%%%%%%%%%%%%%%%%%%%%%%%%%%%%%%%%%%%%%%%%%%%%%%%%%%%%%%%%%%%%%%%%%%%
\begin{frame}{Что не так с современным интернетом?}

\begin{itemize}
\item Вы публикуете мысли, фото, творчество — но все это принадлежит платформе.
\item Алгоритмы решают, кто увидит ваш контент.
\item Правила могут измениться в любой момент, а аккаунт — быть заблокированным.
\item Платформы завалены нейросетевым мусором.
\end{itemize}



\end{frame}
%%%%%%%%%%%%%%%%%%%%%%%%%%%%%%%%%%%%%%%%%%%%%%%%%%%%%%%%%%%%%%%%%%%%%%%%%

%%%%%%%%%%%%%%%%%%%%%%%%%%%%%%%%%%%%%%%%%%%%%%%%%%%%%%%%%%%%%%%%%%%%%%%%%
\begin{frame}{Решение: Indie Web}

\textbf{IndieWeb} — это ориентированная на людей альтернатива «корпоративному интернету».

\vspace{4mm}

\textbf{Основная идея} - это владение собственным доменом

\end{frame}
%%%%%%%%%%%%%%%%%%%%%%%%%%%%%%%%%%%%%%%%%%%%%%%%%%%%%%%%%%%%%%%%%%%%%%%%%



%%%%%%%%%%%%%%%%%%%%%%%%%%%%%%%%%%%%%%%%%%%%%%%%%%%%%%%%%%%%%%%%%%%%%%%%%
\begin{frame}{Преимущества}

\begin{itemize}
\item \textbf{Творческий контроль:} Дизайн вашего сайта именно такой, как вы хотите. Нет навязанных шаблонов или ограничений по количеству символов.
\item \textbf{Владение контентом:} Ваши статьи, фото и идеи остаются вашими. 
\item \textbf{Долговечность:} Ваш сайт существует столько, сколько вы его поддерживаете. Нет закрытия платформ, стирающих годы работы.
\end{itemize}

\end{frame}
%%%%%%%%%%%%%%%%%%%%%%%%%%%%%%%%%%%%%%%%%%%%%%%%%%%%%%%%%%%%%%%%%%%%%%%%%

%%%%%%%%%%%%%%%%%%%%%%%%%%%%%%%%%%%%%%%%%%%%%%%%%%%%%%%%%%%%%%%%%%%%%%%%%
\begin{frame}{Основной принцип:}

\textbf{Основной принцип:} Publish (on your) Own Site, Syndicate Elsewhere (POSSE). Публикуй на своем сайте, а потом распространяй  в соцсети.

\end{frame}
%%%%%%%%%%%%%%%%%%%%%%%%%%%%%%%%%%%%%%%%%%%%%%%%%%%%%%%%%%%%%%%%%%%%%%%%%

%%%%%%%%%%%%%%%%%%%%%%%%%%%%%%%%%%%%%%%%%%%%%%%%%%%%%%%%%%%%%%%%%%%%%%%%%
\begin{frame}{Что нужно чтобы начать?}

\begin{itemize}
\item Создать домен
\item Настроить хостинг
\item Начать вести блог
\end{itemize}

\end{frame}
%%%%%%%%%%%%%%%%%%%%%%%%%%%%%%%%%%%%%%%%%%%%%%%%%%%%%%%%%%%%%%%%%%%%%%%%%


%%%%%%%%%%%%%%%%%%%%%%%%%%%%%%%%%%%%%%%%%%%%%%%%%%%%%%%%%%%%%%%%%%%%%%%%%
\begin{frame}{Neocities}
\textbf{Neocities} — это бесплатный хостинг, позволяющий создать собственный HTML-сайт с нуля, с полным контролем над его функционалом.

\begin{columns}[T]
\begin{column}{0.48\textwidth}
\centering
\includegraphics[width=0.7\textwidth]{images.png}\\
\textbf{Логотип Neocities}
\end{column}
\begin{column}{0.48\textwidth}
\centering
\includegraphics[width=\linewidth]{neocities.png}\\
\textbf{Главная страница}
\end{column}
\end{columns}

\end{frame}
%%%%%%%%%%%%%%%%%%%%%%%%%%%%%%%%%%%%%%%%%%%%%%%%%%%%%%%%%%%%%%%%%%%%%%%%%


%%%%%%%%%%%%%%%%%%%%%%%%%%%%%%%%%%%%%%%%%%%%%%%%%%%%%%%%%%%%%%%%%%%%%%%%%
\begin{frame}{Примеры сайтов}

\begin{columns}[T]
\begin{column}{0.48\textwidth}
\centering
\includegraphics[width=\linewidth]{1.png}\\
\end{column}
\begin{column}{0.48\textwidth}
\centering
\includegraphics[width=\linewidth]{2.png}\\
\end{column}
\end{columns}

\end{frame}
%%%%%%%%%%%%%%%%%%%%%%%%%%%%%%%%%%%%%%%%%%%%%%%%%%%%%%%%%%%%%%%%%%%%%%%%%

%%%%%%%%%%%%%%%%%%%%%%%%%%%%%%%%%%%%%%%%%%%%%%%%%%%%%%%%%%%%%%%%%%%%%%%%%
\begin{frame}{Пример личного блога}
\begin{center}
\includegraphics[width=0.9\textwidth]{3.png} % маленькая и по центру
\end{center}
\end{frame}
%%%%%%%%%%%%%%%%%%%%%%%%%%%%%%%%%%%%%%%%%%%%%%%%%%%%%%%%%%%%%%%%%%%%%%%%%


%%%%%%%%%%%%%%%%%%%%%%%%%%%%%%%%%%%%%%%%%%%%%%%%%%%%%%%%%%%%%%%%%%%%%%%%%
\begin{frame}{}

\begin{columns}[T]
\begin{column}{0.48\textwidth}
\centering
\includegraphics[width=\linewidth]{4.png}\\
\end{column}
\begin{column}{0.48\textwidth}
\centering
\includegraphics[width=\linewidth]{5.png}\\
\end{column}
\end{columns}

\end{frame}
%%%%%%%%%%%%%%%%%%%%%%%%%%%%%%%%%%%%%%%%%%%%%%%%%%%%%%%%%%%%%%%%%%%%%%%%%


%%%%%%%%%%%%%%%%%%%%%%%%%%%%%%%%%%%%%%%%%%%%%%%%%%%%%%%%%%%%%%%%%%%%%%%%%
\begin{frame}{Как найти сайт?}

\begin{itemize}
\item \textbf{Веб-кольцо} — объединение веб-сайтов, при котором каждый участник кольца размещает ссылки на следующего и предыдущего членов кольца. 
\item \textbf{Каталоги по интересам:} Люди вручную составляют списки блогов на определенные темы
\item \textbf{Webmention }— это открытый стандарт и протокол для децентрализованных уведомлений и взаимодействий между сайтами.
\end{itemize}
\end{frame}
%%%%%%%%%%%%%%%%%%%%%%%%%%%%%%%%%%%%%%%%%%%%%%%%%%%%%%%%%%%%%%%%%%%%%%%%%

%%%%%%%%%%%%%%%%%%%%%%%%%%%%%%%%%%%%%%%%%%%%%%%%%%%%%%%%%%%%%%%%%%%%%%%%%
\begin{frame}{Коммьюнити}


\begin{itemize}
\textbf{IndieWebCamp (IWC):} Оффлайн-мероприятия, которые проходят по всему миру (Нью-Йорк, Берлин, Портленд и др.) и онлайн. 
\item \textbf{Homebrew Website Club (HWC):} Регулярные, более камерные онлайн- и оффлайн-встречи.
\item \textbf{Discord}
\item \textbf{GitHub:} Здесь ведется вся техническая работа: обсуждение и разработка стандартов (Webmention, Microformats)
\end{itemize}

\end{frame}
%%%%%%%%%%%%%%%%%%%%%%%%%%%%%%%%%%%%%%%%%%%%%%%%%%%%%%%%%%%%%%%%%%%%%%%%%


%%%%%%%%%%%%%%%%%%%%%%%%%%%%%%%%%%%%%%%%%%%%%%%%%%%%%%%%%%%%%%%%%%%%%%%%%
\begin{frame}{Вывод: Интернет, каким он должен быть}

\begin{center}
\LARGE \textbf{Indie Web — это не о технологиях.\\ Это о \textcolor{myred}{свободе}.}
\end{center}

\vspace{5mm}

\begin{itemize}
    \item Свободе \textbf{владеть} своими текстами, фотографиями, памятью.
    \item Свободе \textbf{выражать} себя без алгоритмических ограничений.
    \item Свободе \textbf{строить} долговечные связи, а не гнаться за лайками.
\end{itemize}

\end{frame}
%%%%%%%%%%%%%%%%%%%%%%%%%%%%%%%%%%%%%%%%%%%%%%%%%%%%%%%%%%%%%%%%%%%%%%%%%





\end{document}