% fancytikzposter.tex, version 2.1
\documentclass{a0poster}
\usepackage{fancytikzposter} 
\usepackage[T2A]{fontenc}
\usepackage[utf8]{inputenc}
\usepackage[english,russian]{babel}
\usepackage{amsmath,amssymb}
\usepackage{graphicx}

\usepackage{relsize}
\usepackage{anyfontsize}




\setmargin{1}
\setblockspacing{1}
\setcolumnnumber{2}
\usetemplate{1}

\setblockspacing{1}
\setcolumnnumber{2}
\definecolor{myblue}{HTML}{008888} 
%\setfirstcolor{myblue}% default 116699
%\setsecondcolor{gray!80!}% default CCCCCC
\setthirdcolor{white!80!black}% default 991111

\usepackage[margin=\margin cm, paperwidth=84.1cm, paperheight=118.9cm]{geometry}

\usepackage{cmbright}
\usepackage[math]{kurier}

\title{Определители: свойства и вычисление}
\author{Довгань М.Д.\\
irasuperstar@yandex.ru\\
   Ярославский государственный университет им. П. Г. Демидова\\
  %\texttt{iliyask@uniyar.ac.ru, igor.maslenikov16@yandex.ru}
}

\begin{document}

\ClearShipoutPicture
\AddToShipoutPicture{\BackgroundPicture}

\noindent
\begin{tikzpicture}
  \initializesizeandshifts

  %% Заголовок в том же формате как в исходном постере
  \ifthenelse{\equal{\template}{1}}{ 
    \titleblock{51}{1}%47
  }{
    \titleblock{51}{1.5}%47
  }
  \addlogo[south west]{(2,2)}{5cm}{yarsu_logor.png}%2


\calloutblock{(0,43)}{(0,43)}{78cm}
{

\textbf{\LARGE  Перестановки}
\vspace{0.5cm}

\raggedright
\fontsize{42}{38}\selectfont % Добавлен \selectfont
\textbf{Определение 1.} \textit{Перестановкой} чисел $1, 2, \ldots, n$ называется их расположение в строку. Иначе говоря, перестановка порядка $n$ есть упорядоченный набор $\alpha := (\alpha_1, \ldots, \alpha_n)$ такой, что все компоненты $\alpha_j$ попарно различны и принадлежат множеству $W_n := \{1, \ldots, n\}$.


\textbf{Определение 2.} С перестановкой $\alpha$ естественным образом связывают понятие \textit{подстановки} порядка $n$, то есть взаимно-однозначного отображения множества $W_n$ на себя. Подстановка, обозначаемая тем же символом $\alpha$, изображается в виде двустрочной таблицы:

\setcounter{equation}{0}
\begin{equation}
\alpha = \begin{pmatrix}
1 & 2 & \ldots & n \\
\alpha_1 & \alpha_2 & \ldots & \alpha_n
\end{pmatrix}.
\end{equation}

Имеется в виду, что $j \mapsto \alpha_j$ или $\alpha_j := \alpha(j)$ (как и всегда для функций натурального аргумента).

\textbf{Определение 3.} На множестве всех подстановок порядка $n$ можно рассмотреть операцию \textit{суперпозиции} $\circ$, определяемую, как обычно, равенством:
\begin{equation}
\alpha \circ \beta(j) := \alpha(\beta(j)), \quad j = 1, \ldots, n.
\end{equation}

}






\plainblock[0]{($(-20,17.0)$)}{38}{\LARGE Свойства перестановок}
{

\fontsize{42}{38}\selectfont % Тот же размер шрифта% Увеличивает расстояние между абзацами

\textbf{1\textdegree.} Количество всех перестановок порядка $n$ равно $n!$.

\setlength{\parskip}{1.0em}

\textbf{2\textdegree.} Любая транспозиция меняет чётность перестановки.

\textbf{3\textdegree.} Число чётных перестановок порядка $n > 1$ совпадает с числом нечётных и равно $n!/2$.

\textbf{4\textdegree.} Если $\beta = \alpha^*$, то $\beta^* = \alpha$. Иначе говоря, $(\alpha^*)^* = \alpha$.

\textbf{5\textdegree.} $s(\alpha^*) = s(\alpha)$.
}


\plainblock[0]{($(20,17.0)$)}{40}{\LARGE Определители} %
{

\fontsize{42}{38}\selectfont
\textbf{Определение 4.} \textit{Определителем} квадратной матрицы $A = (a_{ij})$ порядка $n$ называется число:

\setcounter{equation}{2} % Устанавливает счетчик на 3
\begin{equation}
|A| = \det(A) := \sum_{\alpha} (-1)^{s(\alpha)} a_{1\alpha_1} a_{2\alpha_2} \ldots a_{n\alpha_n},
\end{equation}
% Уравнение получит номер 4 (3+1)
где суммирование осуществляется по множеству всех перестановок порядка $n$.
}

\plainblock[0]{($(20,1.0)$)}{40}{\LARGE Свойства определителя}
{

\fontsize{42}{38}\selectfont

\textbf{1\textdegree.} Определитель линеен по каждой строке.
\setlength{\parskip}{1.0em}

\textbf{2\textdegree.} При умножении строки на число определитель умножается на это число.

\textbf{3\textdegree.} Определитель с нулевой строкой равен нулю.

\textbf{4\textdegree.} Определитель с двумя одинаковыми строками равен нулю.

\textbf{5\textdegree.} Определитель не меняется при прибавлении к строке линейной комбинации других строк.

\textbf{6\textdegree.} При перестановке двух строк определитель меняет знак.

\textbf{7\textdegree.} Определитель не меняется при транспонировании: $|A^T| = |A|$.

\textbf{8\textdegree.} Определитель треугольной матрицы равен произведению элементов главной диагонали.
}

\plainblock[0]{($(-20,-6.0)$)}{38}{\LARGE Правило Крамера}
{

\fontsize{42}{38}\selectfont
\setlength{\parskip}{1.0em}

\textbf{Теорема 1.} Система линейных уравнений $Ax = b$ является определённой тогда и только тогда, когда $|A| \neq 0$.

\textbf{Следствие 1.} Однородная система $Ax = 0$ имеет ненулевое решение $\Leftrightarrow |A| = 0$.

\textbf{Следствие 2.} $|A| = 0 \Leftrightarrow$ строки (и столбцы) $A$ образуют линейно зависимую систему.

\textbf{Теорема 2 (Правило Крамера).} Пусть $d := |A| \neq 0$. Тогда единственное решение системы $Ax = b$ имеет вид:
\setcounter{equation}{3}
\begin{equation}
x_i = \frac{d_i}{d}, \quad i = 1, \ldots, n,
\end{equation}
где $d_i$ — определитель, получающийся из $d$ заменой $i$-го столбца на столбец свободных членов.

\textbf{Замечания:}
\begin{itemize}
\item Если $|A| = 0$, но хотя бы один $d_i \neq 0$, то система несовместна
\item Если $|A| = d_1 = \ldots = d_n = 0$, система может быть как несовместной, так и неопределённой
\end{itemize}
}


\plainblock[0]{($(0,-50)$)}{78}{\LARGE Список литературы} % -31.25
  {
        \fontsize{42}{38}\selectfont
\begin{enumerate}
\bibitem{S1}
{\it Невский М.В.} Лекции по алгебре : Учеб. пособие / Яросл. гос. ун-т. Ярославль, 2002. 265 с. ISBN 5-8397-0202-1
\end{enumerate}
    
    }
\end{tikzpicture}
\end{document}